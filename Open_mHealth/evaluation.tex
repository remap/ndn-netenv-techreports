\section{Evaluation}

At this early stage, we evaluate the pilot application design by discussing whether the affordances of the NDN architecture, libraries, and deployment scenarios make it easier (or possible at all) to meet the requirements of the application. To do so, we employ Green \& Petre's \emph{cognitive dimensions} framework~\cite{green1996usability} to compare a possible IP-based approach with the approach taken here. These dimensions are ``descriptions of the artifact-user relationship, intended to raise the level of discourse.''~\cite{Green1989} They do not provide a comprehensive evaluation framework, merely a starting point for discussion.  

Dimensions of evaluation that we will consider are:
%% TODO: Provide short descriptions
\begin{itemize}
\item Abstraction gradient
\item Closeness of mapping
\item Consistency
\item Diffuseness
\item Error-proneness
\item Hidden dependencies
\item Premature commitment
\item Progressive evaluation
\item Role expressivenesss
\item Secondary notation
\item Viscosity
\item Visibility
\end{itemize}

We also separate discussion of the prototype application from the deployed system of NDN (libraries, testbed, forwarder, etc.) and the architecture itself, following John Wroclawski's suggestion at the 2013 NSF FIA PI Meeting.~\footnote{``All hat, no answers: Some issues related to the evaluation of architecture.'' John Wroclawski, NSF FIA PI Meeting, March, 2013. \url{http://www.nets-fia.net/Meetings/Spring13/FIA-Arch-Eval-JTW.pptx}}

\subsection{Architecture}

Fundamental capabilities of the NDN architecture vs. the IP architecture. 

\subsection{System}

Codebase and testbed. 

\subsection{Prototype}

Pilot application. 



