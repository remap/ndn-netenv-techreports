\section{Related Work}

\subsection{Previous work by the NDN team}

Within the participatory sensing application area of the previous NDN
FIA project, we extended the concept of a host-centric ``personal data
vault'' developed at UCLA and USC~\cite{mun2010personal} to create a
geographically distributed \emph{personal data cloud (PDC)} using NDN.
This prototype went through three iterations reflecting increasing
understanding of how to develop applications using NDN.   The first
version implemented data collections, key management, storage, data
transfer and authentication/setup phases in a way largely analogous to
TCP/IP based applications.  A second revision integrated the PDC
architecture into a deployed participatory sensing application at the
Center for Embedded Networked Sensing, called
Ohmage~\cite{ramanathan2012ohmage}.
%% It included the development of
%mappings between the relational data store used for Ohmage and the PDC
%data collection model. 
The most recent revision transitioned to use the
new Sync primitive for transferring content between entities, and removes
much of the session-like semantics initially
present\footnote{https://github.com/remap/PDC-SYNC}. This experience
will inform work with the Open mHealth team as they continue to develop
pilot applications using Ohmage and similar platforms. With the emergence
of the Sync primitive and our recently developed
ChronoSync synchronization library~\cite{Afanasyev13:CHRONOSYNC}, as well as lighter-weight mobile client
options based on NDN.JS~\cite{ndn-js-NOMEN}, we plan to explore end-to-end applications data dissemination via NDN.


\subsection{Suggested reading}

Estrin, Deborah, and Ida Sim. "Open mHealth architecture: an engine for health care innovation." Science 330.6005 (2010): 759-760.

Diabetes Case Study document from openmhealth.org 

Hicks, John, et al. ohmage: An open mobile system for activity and experience sampling. CENS Technical Reports 100: 1–25, 2010.

Kang, J., Shilton, K., Estrin, D., Burke, J., Hansen, M. "Self-surveillance privacy." Iowa L. Rev. 97 (2011): 809.

\subsection{Motivating Models}

\subsubsection{Open mHealth}
Ohmage, etc.

\subsubsection{Lifestreams}

``Smartphones can capture diverse spatio-temporal data about an individual; including both intermittent self-report, and continuous passive data collection from onboard sensors and applications. The resulting personal data streams can support powerful inference about the user's state, behavior, well-being and environment. However making sense and acting on these multi-dimensional, heterogeneous data streams requires iterative and intensive exploration of the datasets, and development of customized analysis techniques that are appropriate for a particular health domain.

``Lifestreams is a modular and extensible open-source data analysis stack designed to facilitate the exploration and evaluation of personal data stream sense-making. Lifestreams analysis modules include: feature extraction from raw data; feature selection; pattern and trend inference; and interactive visualization. The system was iteratively designed during a 6-month pilot in which 44 young mothers used an open-source participatory mHealth platform to record both self-report and passive data about their diet, stress and exercise. Feedback as participants and the study coordinator attempted to use the Lifestreams dashboard to make sense of their data collected during this intensive study were critical inputs into the design process. In order to explore the generality and extensibility of Lifestreams pipeline, it was then applied to two additional studies with different datasets, including a continuous stream of audio data, self-report data, and mobile system analytics. In all three studies, Lifestreams' integrated analysis pipeline was able to identify key behaviors and trends in the data that were not otherwise identified by participants.''
\cite{hsieh2013lifestreams}
%%https://www.dropbox.com/s/l0721457faswj5k/a5-hsieh.pdf


\subsection{Other solutions}

HumanAPI
Also OAuth 2.0
Apparently similar objectives to Open mHealth

Ginger.io
Platform for predictive modeling
