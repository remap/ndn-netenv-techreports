\section{Design of NDNEx}

\paragraph{Basis for Design}
\begin{itemize}
\item Review and evaluation of Ohmage reference application and open mHealth schema.
\item Review high level motivation of Estrin \& Sim, 2010 and other references above. 
\item Review case studies, including those on Ohmage website and in the Appendices. 
\item Review Past CENS/UCLA participatory sensing research in activity classification,self-surveillance privacy, mobile phone based data collection.
\end{itemize}

\paragraph{Design Goals} 

\begin{itemize}
\item \textbf{Interoperable, Internet-inspired data exchange} as the backbone of the application ecosystem. 
\item \textbf{Thin waist of open data interchange standards} that will enable an ecosystem of sensing, storage, analysis, and user interface components to support medical discovery and evidence-based care 
\item Market-supported, patient-centered landscape of innovative health applications
\item \textbf{Patient-controlled, privacy-aware data exchange} across device, component, and application boundaries
\item Imported from Open mHealth:
    \begin{itemize}
    \item Data-centric rather than service-centric interoperability. $\Rightarrow$ \textbf{Focus on data namespace design.}
    \item Distributed architecture of Capture, DSU, DPU, DVU. $\Rightarrow$ \textbf{Implement data flow approach in NDN.}
    \item End user focus (not hospitals, doctors, etc.) $\Rightarrow$ \textbf{Consumer app deployment scenario.}
    \item User-centric privacy approach. $\Rightarrow$ \textbf{Need to inform user of choices, data flow.}
    \item Encrypted communications. $\Rightarrow$ \textbf{Encryption-based access control, name encryption.}
    \item Mobile publishing. $\Rightarrow$ \textbf{Use as our driver to solve this oft-cited challenge.}
    \end{itemize}
\item In contrast to Open mHealth:
    \begin{itemize}
    \item REST/HTTP is not used.  $\Rightarrow$ \textbf{Move away from RPC call model} and carrying state in Interests, towards data dissemination.
    \item Host-based endpoints for services. $\Rightarrow$ \textbf{Focus on data dissemination model} and NFN style \textbf{distributed processing}. 
    \item OAuth authentication $\Rightarrow$ Need new identity / authentication approach. 
    \item Single storage ``location'' $\Rightarrow$ \textbf{Distributed, ``personal'' repositories}. 
    \end{itemize}
\end{itemize} 

\paragraph{mHealth Reality Check}
A reality check for all mHealth applications, from the PLOS Medicine Editors. "A reality checkpoint for mobile health: three challenges to overcome." PLoS Medicine 10.2 (2013).
%% TODO: Add to cites
\begin{itemize}
\item \textbf{Are your systems interoperable?}
Estrin \& Sim in Science, 2010.  Open mHealth. 
\item \textbf{Are you using open standards?}
WHO, 2013.  eHealth unit. 
\item \textbf{How will you evaluate?}
Greenhalgh et al. in BMC Med Res. Methodology, 2011.
Realist and meta-narrative evidence synthesis.
\end{itemize}

\subsection{NDNEx Ecosystem components}

First, we introduce the components of the NDNEx application ecosystem, then we discuss overall design strategies for trust, naming, etc. 

We use Ohmage client and server as references applications, and a conceptual design based on  
PEIR, the personal environmental impact report, as a platform for participatory sensing systems research. Mun, M., Reddy, S., Shilton, K., Yau, N., Burke, J., Estrin, D., ... \& Boda, P. (2009, June). Proc. ACM Mobisys 2009. 

User interacts with three pieces:
\begin{itemize}
\item Mobile capture application
\item Website for visualization / review
\item Identity manager (newly envisioned mobile component)
\end{itemize}

\begin{figure}
\begin{center}
\includegraphics[width=.8\textwidth]{figures/NDNEx-App-Architecture-01}
\caption{Draft NDNEx ecosystem design.}
\label{fig:ndnex-ecosystem}
\end{center}
\end{figure}

%%%%%%%%%%%%%%

\subsubsection{Ohmage-NDN Mobile Activity Capture}

\paragraph{Responsible team members}
Anyang University.

\paragraph{Function}
Capture location and activity information, publish. 

\paragraph{Approach}
 
Initially, we will only support Android. Port NFD, tools. Provide NDN-CCL and NFD support for Android. 

Mobile capture application is one of two primary user interfaces.  

Initial analysis of Ohmage mobile client communication completed by Prof. Euihyun Jung�s group from Anyang Univ.   

From Basel: Today, we had an internal presentation from the Psychology Dept which explained their use of the Ohmage system that is running in Basel. Although NDNex comes too late for being applied in ongoing projects, these projects are a good source of insights about user requirements for our future NDNex system, especially the aspect of data sovereignty.

\paragraph{Building Block: Ohmage \& Mobility}

\url{http://ohmage.org/}
We are going to use the Ohmage Android client (including the Mobility module) with only the in-app storage and communication changed to NDN. See (Tangmunarunkit, H., et al., 2014) and also the papers on the Ohmage website. Note that Mobility generates activity classified data that may not require the Activity Classification DPU in the initial version.

ohmage is an open-source participatory sensing technology platform. It supports: 1) expressive project authoring; 2) mobile phone-based data capture through both inquiry-based surveys and automated data capture as well as temporally and/or spatially triggered reminders, 3) data visualization and real-time feedback; privacy respecting data management; and 4) extensible data exploration. 

Tangmunarunkit, H., et al. "Ohmage: A General and Extensible End-to-End Participatory Sensing Platform,� ACM Trans. on Intelligent Systems and Technology (in submission), UCLA CS Technical Report 140015. (Used in 20 projects.)  \url{http://web.ohmage.org/~hongsudt/pub/ohmage_ucla_140015.pdf}

%%%%%%%%%%%%%%

\subsubsection{Personal Data Repository}

\paragraph{Responsible team members}
Design led by Jianxun in Dan Pei�s group at Tsinghua.

\paragraph{Function}

Provide Open mHealth Data Storage Unit (DSU) functionality.  (Also, ``personal data vault''.) 
Personal in that it is controlled by the end user.  NOT necessarily a home repository.  More realistically, a storage service with fiduciary responsibility to protect the data of the end-user per Kang et al. 

\paragraph{Approach}

One or more new NDN repo designs supporting a hierarchy of storage needs: mobile device, user private repository, temporary processing storage.
Storage at:
\begin{itemize}
\item The mobile device itself.
\item A personal data repository (which may be distributed).
\item Temporary storage for processing and visualization components. 
\end{itemize}

Current plan:  implement in Java using jNDN, for Android support.  

\paragraph{Reference: Open mHealth Data Storage Unit (DSU) Design}

\url{https://github.com/openmhealth/developer/wiki/DSU-Overview}

The Open mHealth DSU (Data Storage Unit) API Specification is an open specification for unified information sharing across disparate data streams. The idea is simple: create an easy-to-understand set of APIs that allow siloed data stores to share information. Third-party applications that understand this API specification can then create a single set of tools to access data across any of the servers.

\paragraph{Building Block: Personal Data Vault}
Reference Derek's work here? 

Mun, Min, et al. "Personal data vaults: a locus of control for personal data streams." Proceedings of the 6th International Conference. ACM, 2010.http://remap.ucla.edu/jburke/publications/Mun-et-al-2010-Personal-Data-Vaults.pdf

Kang, J., Shilton, K., Estrin, D., Burke, J. "Self-surveillance privacy." Iowa L. Rev. 97 (2011): 809.http://escholarship.org/uc/item/1jk8b2q1.pdf


%%%%%%%%%%%%%%

\subsubsection{Distributed Processing Blocks}

\paragraph{Responsible team members}
Basel

\paragraph{Function}
Goal here is to have a few representative components implemented using NFN-style approach, not exactly the processing blocks listed above, necessarily.   Start with GeoFencing, as activity classification is currently handled in the Mobility portion of Ohmage.  But, could also consider other application-specific processing ideas. 

Initially, focusing on location-based triggers (geofencing) - to trigger location-based content.

\paragraph{Approach}

Ideally, provide composable data flow inspired by Google Cloud Dataflow, Apache Spark, etc.  

Web-based front end using NDN-JS with access to geofenced location information, providing location-specific content back to the mobile user.

Related to vehicular networking work. 

\paragraph{Reference: Open mHealth Data Processing Unit (DPU) Design}

\url{https://github.com/openmhealth/developer/wiki/Open-mHealth-and-Data-Processing}

DPUs are stateless modules that input and output data. They are designed to be embedded in other software or called remotely. They do not produce anything directly visible, but are the brains and muscles of an application. The concept of a DPU is inspired by the Unix Philosophy of creating small functional tools that can be chained and reused, rather than a single large application.

\paragraph{Building Block: Named Function Networking }

\url{http://www.named-function.net/}

Names serve to access and invoke functions, which incidentally can produce passive content once it is needed. New questions arise from this point of view, namely how the network organizes the flow of functions, which brings us squarely into active networking turf. 

\subsubsection{Content Source: Trails Database}

\url{http://archinect.com/news/article/111897927/tour-los-angeles-history-with-ucla-s-new-interactive-urban-trail-app}

The LASHP Trails Mobile Website gives residents and visitors to Northeast Downtown Los Angeles site-specific access to a dynamic combination of historic information and health-related activities along urban trails starting and ending at the Los Angeles State Historic Park. 

%%%%%%%%%%%%%%

\subsubsection{Visualization Interface}

\paragraph{Responsible team members}
UCLA REMAP

\paragraph{Function}
Provide basic visualization of fitness data.

\paragraph{Approach}
Start with Ohmage web front end
NDN-JS and D3
Web-based front end using NDN-JS with access to geofenced location information, providing (for example) running trail visualization.
Perhaps use many GPX format visualizers.  E.g., \url{http://flowingdata.com/2014/02/05/where-people-run/}

\paragraph{Reference: Lifestreams Dashboard}
\cite{hsieh2013acm}

\paragraph{Building Block: Analytics / Presentation: Ohmage Front-end for Mobilize}

\url{https://wiki.mobilizingcs.org/app/web}

And Lifestreams?
Web-based front end using NDN-JS to access derived data without location information.
Examples: http://quantifiedself.com/fitbit/   

The web frontend (powered by the ohmage project) is used to provide students secure access to their data. It supports secure login, campaign management, data management and basic campaign monitoring and visualization. The students can review and share their data to the growing data set collected by their class. The web frontend can also be used to discover the answers to basic statistical inferences in real-time as data is being collected. When data collection is complete, the web frontend allows for easy exporting of the data to a more thorough statistical analysis tool. 



%%%%%%%%%%%%%%

\subsubsection{Identity Manager}
    
\paragraph{Responsible team members}
UCLA IRL (Yingdi)? and REMAP (Dustin)?
  
%%%%%%%%%%%%%%
  
\subsection{Trust and security}

\paragraph{Responsible team members}
U. Michigan, UCLA IRL, UCLA REMAP

Replacing Oauth2 for distributed processing is critical
One
entity (here, the `` user") maintains multiple publishers whose data are
consumed by many services with varying levels of access based on the: type
of data (as expressed in the name), level of granularity, and date/time
when the data was produced. ABE seems a good fit here, is it viable for
practical experimentation or do we need a directory-of-symmetric-keys or
some other approach?  If not ABE, where should we begin for our approach?
Are there existing best practices?


\subsubsection{Trust model} 
Leverage PKI as deployed
Would like to start from the same tools being used for user/gateway trust management in NDN testbed for this application.  
How to take advantage of the NDN signing infrastructure?
Certificates and delegation of authority.

\subsubsection{Identity}
Each processing block in this diagram may come from a different service provider. 
User may have different identity per service (or at least per flow). 
Each step tends to generate derived data that must also be stored and may not be associated with the original identity.  

We are exploring the idea of an ``identity manager", an application
manages the certificates (identities) that an individual uses to interact
with the various services involved in this application. Are there good
examples of \emph{user interfaces} for identity management already?  In fact,
pointers to state-of-the-art in end-user interfaces for security decision
making would be helpful. Alex doesn't think there are many. 

\subsubsection{Integrity} 

\subsubsection{Confidentiality}

Principal of minimum information. 

Access control in "data flow" model for communication between processing components, replacing Oauth (with Tsinghua and UCLA).  Extensions to support epidemiological studies incorporating semi-anonymized opt-in data across large populations. 

We will need to come up with an approach for name encryption for
this environment, in a way that still enables applications to operate on
the namespace--perhaps without having to be concerned with
decryption--once it has been decrypted and/or de-encapsulated.


\subsubsection{Data flow model support}
We imagine a data flow like model for this system:
[Publisher]->[Processing]->[Processing]->[Visualization], with each []
block being owned by a different entity and an objective to leak the
minimum amount of context to the processing components.  I'm not sure we
understand how to handle authentication /  access control of the
intermediate processing blocks, to each other and to the source/sink of
the data.  Where can we look for best practices for security in current
data flow architectures?

%%%%%%%%%%%%%%
  
\subsection{Naming}

\paragraph{Responsible team members}
UCLA REMAP, UCLA IRL


\subsubsection{Data}
Personal health data (and metadata) namespace and repository design � focusing on support for physical activity data in the first round.  
What schema? Initially, try direct mapping of Open mHealth schema to names


\textbf{Basis of design}: Open mHealth Physical Activity data schema\footnote{\url{http://www.openmhealth.org/developers/schemas/#physical-activity} and \url{http://bioportal.bioontology.org/ontologies/SNOMEDCT?p=classes&conceptid=68130003}}, as well as other schemas from Open mHealth as needed. 

\input{physical-activity-schema-json}

How to separate this into namespace and data?
Figure~\ref{fig:ndnex-namespace}

JSON payload. 

\begin{figure}
\begin{center}
\includegraphics[width=1\textwidth]{figures/ndnex-name-top-01}
\caption{NDNEx Namespace, version 1.}
\label{fig:ndnex-namespace}
\end{center}
\end{figure}

\subsubsection{Certificates}

\subsubsection{Processing}
Borrow ideas from Named Function Networking concept for distributed processing

%%%%%%%%%%%%%%
  
\subsection{Storage}

\paragraph{Responsible team members}
UCLA IRL, Tsinghua, UCLA REMAP

A distributed network of repos replaces the ecosystem of DSUs envisioned in the Open mHealth TCP/IP architecture. 

(Hierarchical network of repositories, similar to BAS/BMS, including both personal repositories, service provider backups for personal data, and aggregated "anonymized" stores). 

Write access control

A mechanism for delegating authority to publish into a repository is
necessary. 

%%%%%%%%%%%%%%
  
\subsection{Routing \& Forwarding}

\paragraph{Responsible team members}
UCLA IRL

Mobile publishing support. NDNS?

\begin{figure}
\centering
\subfigure[Mobile publisher connected to ``home'' hub, which directly routes its publishing prefix.]{
\includegraphics[width=0.45\columnwidth, keepaspectratio=true]{figures/publisher-mobility-a}
}
\subfigure[Mobile publisher connected to another hub, which does not directly route its prefix.]{
\includegraphics[width=0.45\columnwidth, keepaspectratio=true]{figures/publisher-mobility-b}
}
\caption{Mobile publisher scenario.}
\label{fig:mobilepublisher}
\end{figure}


