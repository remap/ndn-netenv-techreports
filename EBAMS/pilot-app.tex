\section{Pilot Application: NDN-EBAMS}


Our pilot applications:  \textbf{NDN-EBAMS} deployment at UCLA. 

Build an NDN-based collection, storage, and query
system for real UCLA Facilities Management data coming from the Siemens
building monitoring system. (Target ~10k points at up to 1Hz in 2016.)  We
are initially focused on \emph{read-only} access to sensing data.

Scope of the data: the 800 or so points of monitoring that we just got
access to generate 24M rows per year and cover a few buildings and a few
data types.  We hope to scale to 10k pts by the end of 2016.  Points are
monitored no faster than 1 Hz.  We get changes only, so time stamps are
irregular.



\subsection{Application Requirements}

%% Time-series
%%

%% Database queries

\subsubsection{Naming and application design}
\begin{itemize}
\item Reflect system and physical world knowledge in the naming 
\item Simplify application development (seeking evaluation approaches)
\end{itemize}

\subsubsection{Trust and security}
Fit NDN architectural mechanisms into security requirements
\begin{itemize}
\item Base on real administrative organization at UCLA
\item Trust model? Hierarchical may work but may be in a different namespace from the data  (see Wentao’s work so far) 
\item Validating / scaling up encryption based access control approach using actual use cases.
\end{itemize}

\subsubsection{Storage in the network}
\begin{itemize}
\item Support the basic reporting requirements of the campus operators
\item Repository design to support SQL-style queries by UCLA Facilities Management on top of distributed repositories in the style of Apache Hive. 
\end{itemize}
