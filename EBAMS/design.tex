\section{Design}

Develop repository design and basic trust approach for offline report generation and online real-time viewing of the data as typically used by Fac Mgmt. 

\paragraph{Challenges}
\begin{itemize}
\item Namespace design to tackle overlapping roles of names in application data access, routing, device deployment, security, etc. 
\item Storage/repository design that provides familiar query interfaces.
\item Discovery and bootstrapping that is easier and more secure than IP. 
\item Trust model development and implementation (a good problem to have). 
\item Efficient crypto performance and group / multi-cast cryptographic challenges. 
\item Low-frequency, high-importance notifications (alarms).
\end{itemize}

\paragraph{Approach.}

\begin{itemize}
\item A hierarchical namespace for sensor data, devices, and users that embodies intrinsic relationships in BMS applications and can be used directly for data delivery;
\item Bootstrapping and ongoing management of device configuration that is simpler, more scalable and more robust than IP solutions;
\item Per-packet signatures, applied immediately after data acquisition, that enable  straightforward verification of data authenticity and provenance;
\item Strong security through cryptographic signing to verify data authenticity and provenance;
\item Encryption-based access control to sensor data, with data encrypted immediately after its acquisition, rather than relying on encrypted channels; 
\item Strong privacy through encryption-based access control to protect sensing data, without reliance on physical/logical network isolation.
\item Identity-based authentication using NDN's (still developing) security primitives; 
\item Scalable user and privilege management to support enterprise-wide deployments. 
\end{itemize}




\subsection{NDN-EBAMS Application Architecture} 
 
Cite Wentao's paper for full information on design

  
 \subsection{Trust and security}
 
Only focus on read access control. No write access.

Opportunity to compare active services (e.g., broker) with passive. 

\subsubsection{Trust model} 
Leverage PKI as deployed
Would like to start from the same tools being used for user/gateway trust management in NDN testbed for this application.  
How to take advantage of the NDN signing infrastructure?
Certificates and delegation of authority.

\subsubsection{Identity} 

\subsubsection{Integrity} 

\subsubsection{Confidentiality}
Crypto approach.
Groups: NDN-BmS developers at UCLA, UIUC, departmental/campus stakeholders, and testbed users. 
Users:  Users possessing a cert in new system, and anonymous users on the web.
Applications that are not associated with a real-world users but need credentials: aggregators, gateways, background processes, etc. 

There appear to be two important factors to
consider in controlling access to a specific monitoring point or group of
points in our pilot: 1) what department it relates to (not necessarily in the name),
2) what type of data it is (probably in the name).
Each "point" here corresponds to a source of time-series monitoring data.
For UCLA, I would expect there are O(100) "departments", with many
individuals having access for several departments, and O(10) types of
data, for our purposes.

What is the appropriate practical crypto scheme to develop access control
based on these dimensions?  I really like the concept of ABE, as long as
there is some ability to deploy it practically, because it would enable us
to define sensible *named* attributes that correspond directly to
department and data type. A relationship between attribute names and data
names (for key data, etc.) might be useful to simplify things at the
application level.



\subsection{Naming}
\subsubsection{Data}

Discover
Design focuses on the actual deployment
Data sources \& sinks:  sensors, actuators.
Physical space / location will be involved in trust and should be described consistently and considered along with routing / application implications.
Approach to naming/retrieving sample batches. 
Approach to metadata describing sensor feeds. 

Hierarchical data names and simple sensor/control abstraction.
See also: 

Dawson-Haggerty, Stephen, et al. "Enabling green building applications." Proceedings of the 6th Workshop on Hot Topics in Embedded Networked Sensors. ACM, 2010.

Ortiz, Jorge, and David Culler. A system for managing physical data in buildings. Technical Report No. EECS-2010-128, EECS Department, University of California, Berkeley, 2010.

Potentially can use names for:
\begin{itemize}
\item Abstraction (allow application generalization)
\item Organization (by building area, system topology) 
\item Metadata access (consistent approach)
\item Aggregation (hierarchy inferred from names) 
\end{itemize} 



BOSS~\cite{Dawson-Haggerty2013BOSS} follows these references for building metadata. 

[8] Bazjanac, V., et al. HVAC component data modeling using industry foundation classes. In System Simulation in Buildings (2002). 

[30] Liu, X. et al, Requirements for a formal approach to represent information exchange requirements of a self-managing framework for HVAC systems. In ICCCBE (2012). 

[40] Project Haystack. http://project-haystack.org/  


\subsubsection{Certificates}

\subsection{Storage}

Repository for time-series data used in BOSS~\cite{Dawson-Haggerty2013BOSS} is readingdb, \url{https://github.com/stevedh/readingdb}.
%% TODO: Review BOSS paper and incorporate here. 
Their simplification of repositories is about data type more than query type:  focus on time series data from points. 
MySQL not used because of expensive insert and poor scaling with large number of leaf keys.
Low latency application interface for accessing the large repository of data at different granularity, a selection language, and a data transformation language. Enabling the construction of a pipeline of operators to the retrieved data. 

Repository requirements to support sql-style access

Designing and deploying hierarchical repositories for sensor data (from sensor module to panel to building to campus scale) that provide quick access to fresh data and efficient, redundant long-term storage. 

More activity on repo-ng needed? 

Reasonable performance on a variety of device classes  running as a component in a data producer and as a centralizer of data. 

Write performance for main repo:  20,000 samples @ 1Hz on an ongoing basis.  

Mechanism to export / move data offline. 

Need a simple way to watch time-series prefixes. 

Restore sync support for more complex namespaces. 

\subsection{Routing \& Forwarding}

Limit packets inside and outside? 

\subsection{Communication}
Alarms
Efficient support for specific types of communication patterns found in these networks, like reliable notification of rare but critical events. 


\subsection{User Interface}
\subsubsection{Website}
