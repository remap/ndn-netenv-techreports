\section{Design}

Develop repository design and basic trust approach for offline report generation and online real-time viewing of the data as typically used by Fac Mgmt. 

\paragraph{Challenges}
\begin{itemize}
\item Namespace design to tackle overlapping roles of names in application data access, routing, device deployment, security, etc. 
\item Storage/repository design that provides familiar query interfaces.
\item Discovery and bootstrapping that is easier and more secure than IP. 
\item Trust model development and implementation (a good problem to have). 
\item Efficient crypto performance and group / multi-cast cryptographic challenges. 
\item Low-frequency, high-importance notifications (alarms).
\end{itemize}

\paragraph{Approach.}

\begin{itemize}
\item A hierarchical namespace for sensor data, devices, and users that embodies intrinsic relationships in BMS applications and can be used directly for data delivery;
\item Bootstrapping and ongoing management of device configuration that is simpler, more scalable and more robust than IP solutions;
\item Per-packet signatures, applied immediately after data acquisition, that enable  straightforward verification of data authenticity and provenance;
\item Strong security through cryptographic signing to verify data authenticity and provenance;
\item Encryption-based access control to sensor data, with data encrypted immediately after its acquisition, rather than relying on encrypted channels; 
\item Strong privacy through encryption-based access control to protect sensing data, without reliance on physical/logical network isolation.
\item Identity-based authentication using NDN's (still developing) security primitives; 
\item Scalable user and privilege management to support enterprise-wide deployments. 
\end{itemize}


\subsection{NDN-EBAMS   Application Architecture} 
 
 Cite Wentao's paper for full information on design
 
 
 
 
\subsection{Naming}
Discover
Design focuses on the actual deployment
Data sources \& sinks:  sensors, actuators.
Physical space / location will be involved in trust and should be described consistently and considered along with routing / application implications.
\subsection{Storage}
Repository requirements
\subsection{Trust and security}
Crypto approach.
Groups: NDN-BmS developers at UCLA, UIUC, departmental/campus stakeholders, and testbed users. 
Users:  Users possessing a cert in new system, and anonymous users on the web.
Applications that are not associated with a real-world users but need credentials: aggregators, gateways, background processes, etc. 

\subsection{Communication}
Alarms
